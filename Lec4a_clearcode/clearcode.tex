\documentclass[12pt,t]{beamer}
\usepackage{graphicx}
\setbeameroption{hide notes}
\setbeamertemplate{note page}[plain]
\usepackage{listings}

../Lec4a_clearcode/header.tex

%%%%%%%%%%%%%%%%%%%%%%%%%%%%%%%%%%%%%%%%%%%%%%%%%%%%%%%%%%%%%%%%%%%%%%
% end of header
%%%%%%%%%%%%%%%%%%%%%%%%%%%%%%%%%%%%%%%%%%%%%%%%%%%%%%%%%%%%%%%%%%%%%%

\title{Writing clear code}
\author{\href{https://kbroman.org}{Karl Broman}}
\institute{Biostatistics \& Medical Informatics, UW{\textendash}Madison}
\date{\href{https://kbroman.org}{\tt \scriptsize \color{foreground} kbroman.org}
\\[-4pt]
\href{https://github.com/kbroman}{\tt \scriptsize \color{foreground} github.com/kbroman}
\\[-4pt]
\href{https://twitter.com/kwbroman}{\tt \scriptsize \color{foreground} @kwbroman}
}

\begin{document}

{
\setbeamertemplate{footline}{} % no page number here
\begin{frame}{}

\vspace{18pt}

\figh{Figs/i_dont_get_your_code.jpg}{0.8}

\vspace{18pt}

\hfill {\tt \footnotesize \lolit \href{http://geekandpoke.typepad.com/geekandpoke/2008/02/the-art-of-prog.html}{geekandpoke.typepad.com}}

\note{It's funny because it's true.
}

\end{frame}
}


{
\setbeamertemplate{footline}{} % no page number here
\frame{
  \titlepage

\note{Clear code is more likely to be correct.

  Clear code is easier to use.

  Clear code is easier to maintain.

  Clear code is easier to extend.
}
} }




\begin{frame}{Basic principles}

\vspace{18pt}

\bi
\item Code that works
    \bi
    \item[] No bugs; efficiency is secondary (or tertiary)
    \ei
\item Readable
    \bi
    \item[] Fixable; extendible
    \ei
\item Reusable
    \bi
    \item[] Modular; reasonably general
    \ei
\item Reproducible
    \bi
    \item[] Re-runnable
    \ei
\item Think before you code
    \bi
    \item[] More thought $\implies$ fewer bugs/re-writes
    \ei
\item Learn from others' code
    \bi
    \item[] R itself; key R packages
    \ei
\ei

\note{Our first goal should be to get the right answer.

  But we also want to write code that we (and others) will be able to
  use again. Efficiency is way down on the list of needs.

  Write clearly, and re-write for greater clarity.

  Break things down into small, re-usable functions, written in a
  general way, but not {\nhilit too} general. You want to actually
  solve a particular problem.

  Step 1: stop and think.

  Step 2: write re-usable functions rather than a script.

  By reproducible/re-runnable, I mean: don't have a bunch of code from
  which you copy-and-paste in a special way.
}
\end{frame}


\begin{frame}[c]{}

\vspace*{30mm}

\centerline{\large
Write programs for people, not computers}

\vspace{30mm}

\hfill {\footnotesize \href{http://journals.plos.org/plosbiology/article?id=10.1371/journal.pbio.1001745}{Wilson
  et al. (2014) PLoS Biol 12:e1001745}}


\note{
  The computer doesn't care about comments, or style, or the naming of
  things. But users and collaborators (and yourself six months from
  now) will.
}
\end{frame}

\begin{frame}[fragile,c]{Break code into small functions}

\begin{lstlisting}
get_grid_index <-
function(vec, step)
{
  grid <- seq(min(vec), max(vec), by=step)
  index <- match(grid, vec)

  if(any(is.na(index)))
    index <- sapply(grid, function(a,b) {
      d <- abs(a-b)
      wh <- which(d==min(d))
      if(length(wh)>1) wh <- sample(wh, 1)
      wh
      }, vec)

  index
}
\end{lstlisting}


\note{This is a bit of code that was originally part of a larger
  function.

  Separated as a function, the code is more likely to be reused. And
  the smaller and more focused the function, the more likely you'll
  use it again.

  Avoid having scripts that are long streams of code.
  Write functions. Give each function a single, focused
  task. Break long functions into pieces.

  Comment on the input and output; ideally, these
  are obvious from the names.

  What might be improved about this function?
}
\end{frame}


\begin{frame}[fragile,c]{Break code into small functions}

\begin{lstlisting}
sampleone <-
function(vec)
  ifelse(length(vec)==1, vec, sample(vec, 1))

get_grid_index <-
function(vec, step)
{
  grid <- seq(min(vec), max(vec), by=step)
  index <- match(grid, vec)

  if(any(is.na(index)))
    index <- sapply(grid, function(a,b) {
        d <- abs(a-b)
        sampleone(which(d == min(d)))
      }, vec)

  index
}
\end{lstlisting}


\note{Ideally, a function does only one thing.

  Pulling out the {\ttsm sampleone} code as a separate function makes
  the body of {\ttsm get\_grid\_index} a bit simpler.

  Is the {\ttsm ifelse} line clear? Would it be better to just write
  that function with {\ttsm if} and {\ttsm else}?

  Could {\ttsm get\_grid\_index} be improved further?
}
\end{frame}


\begin{frame}[fragile,c]{Clarity over efficiency}

\begin{lstlisting}
sampleone <-
function(vec)
  ifelse(length(vec)==1, vec, sample(vec, 1))

get_grid_index <-
function(vec, step)
{
  grid <- seq(min(vec), max(vec), by=step)
  index <- match(grid, vec)

  if(any(is.na(index))) {
    for(i in seq(along=grid)) {
      d <- abs(grid[i] - vec)
      index[i] <- sampleone(which(d==min(d)))
    }
  }

  index
}
\end{lstlisting}


\note{The use of {\ttsm sapply} is a bit confusing and hard to
  follow. I think it's a bit more clear to use a {\ttsm for} loop.

  Whether {\ttsm apply}-type functions are more readable or less
  readable depends on the complexity of the function that is applied
  but also on the experience of the reader.
}
\end{frame}


\begin{frame}[fragile,c]{One last change}

\begin{lstlisting}
sampleone <-
function(vec)
  ifelse(length(vec)==1, vec, sample(vec, 1))

get_grid_index <-
function(vec, step)
{
  grid <- seq(min(vec), max(vec), by=step)
  index <- match(grid, vec)

  missing <- is.na(index)

  if(any(missing)) {
    for(i in which(missing)) {
      d <- abs(grid[i] - vec)
      index[i] <- sampleone(which(d==min(d)))
    }
  }

  index
}
\end{lstlisting}


\note{One last change: The loop really only needs to be over the parts
  of {\ttsm index} that are missing.

  Of course, I should also add a comment for each function, to explain
  the input and output.
  }
\end{frame}




\begin{frame}[fragile,c]{Another example}

\begin{lstlisting}
# rmvn: simulate from multivariate normal distribution
rmvn <-
function(n, mu=0, V=diag(rep(1, length(mu))))
{
  p <- length(mu)

  if(any(dim(V) != p))
    stop("Dimension problem!")

  D <- chol(V)

  matrix(rnorm(n*p),ncol=p) %*% D + rep(mu,each=n)
}
\end{lstlisting}
\note{I'm often wanting to simulate from a multivariate normal
  distribution. It's not hard, but you have to remember: do you do
  {\tt Z \%*\% D} or {\tt D \%*\% Z}? And in any case, it's a lot
  easier to just write {\tt rmvn(n, mu, V)}.

  Note the use of default values.
}
\end{frame}

\begin{frame}[fragile,c]{Further examples}

\begin{lstlisting}
# colors from blue to red
revrainbow <-
function(n=256, ...)
  rev(rainbow(start=0, end=2/3, n=n, ...))



# move values above/below quantiles to those quantiles
winsorize <-
function(vec, q=0.006)
{
  lohi <- quantile(vec, c(q, 1-q), na.rm=TRUE)
  if(diff(lohi) < 0)
    lohi <- rev(lohi)

  vec[!is.na(vec) & vec < lohi[1]] <- lohi[1]
  vec[!is.na(vec) & vec > lohi[2]] <- lohi[2]

  vec
}
\end{lstlisting}

\note{If there's a bit of code that you write a lot, turn it into a
  function.

  I can't tell you how many times I typed
  {\tt rev(rainbow(start=0, end=2/3, n=256))} before I thought to make
  it a function. (And I can't tell you how many times I've typed it
  since, having forgotten that I wrote the function.)

  (And really, I should forget about {\ttsm revrainbow}; such rainbow
  color scales are generally considered a bad idea.)

  I learned about Winsorization relatively recently; it's a really
  useful technique to deal with possible outliers. It's not hard. But
  how much easier to just write {\tt winsorize()}!
}
\end{frame}


\begin{frame}{Writing functions}

\bbi
\item Break large tasks into small units.
  \bi
  \item Make each discrete unit a function.
  \ei
\item If you write the same code more than once, \\
{\hilit make it a function}.
\item If a line/block of code is complicated, \\
{\hilit make it a function}.
\ei

\note{Modularity is really important for readability.

   No long streams of code; rather, a short set of function calls.
}
\end{frame}


\begin{frame}{Don't repeat yourself (or others)}

\bbi
\item Avoid having repeated blocks of code.
\item Create functions, and call those functions repeatedly.
\item This is easier to maintain.
  \bi
  \item If something needs to be fixed/revised, you just have to do it
    the one time.
  \ei
\item Look at others' libraries/packages.
  \bi
  \item Don't write what others have already written (especially if
    they've done it better than you would have).
  \ei
\ei

\note{Don't repeat yourself (DRY) is one of the more important
  concepts for programmers.

  I'm particularly bad at making use of others' code.
}
\end{frame}


\begin{frame}{Don't make things too specific}

\bbi
\item Write code that is a bit more general than your specific data
  \bi
  \item Don't assume particular data dimensions.
  \item Don't forget about the possibility of missing values\\
    (even if {\hilit your} data doesn't have any).
  \item Aim for re-use.
  \ei
\item Use function arguments
  \bi
  \item Don't assume particular data file names
  \item Don't hard-code tuning parameters
  \item R scripts can take \href{http://stackoverflow.com/questions/2151212/how-can-i-read-command-line-parameters-from-an-r-script}{command-line arguments}:\\[2pt]
    \bi
    \item[] {\ttsm Rscript myscript.R input\_file output\_file} \\[2pt]
    \item[] {\ttsm args <- commandArgs(TRUE)}
    \ei
  \ei
\ei

\note{Don't make things too general, either. You want to actually
  solve your particular problem.

  You don't want to have to edit the code for different data.

  We {\nhilit will} write scripts. But make them a bit more general by
  allowing command-line arguments to specify input and output file
  names. You can always include defaults, for your particular
  files. Even better is to have these specified in your {\tt
    Makefile.}  }
\end{frame}



\begin{frame}{No global variables, ever!}

\bbi
\item Don't refer directly to objects in your workspace.
\item If a function needs something, pass it as an argument.
\item (But what about really big data sets?)
\ei

\note{Global variables make code unreadable.

  Global variables make code difficult to reuse.

  It turns out that, when you pass data into a function, R doesn't
  actually make a copy. If your function makes any change to the data
  object, {\nhilit then} it will make a copy. But you shouldn't be
  afraid to pass really big data sets into and between functions.
}
\end{frame}



\begin{frame}{No magic numbers}

\bbi
\item Name numbers and use the names
  \bi
  \item[] {\tt max\_iter <- 1000}
  \item[] {\tt tol\_convergence <- 0.0001}
  \ei
\item {\hilit Even better}: include them as function arguments
\ei

\note{You'll later ask yourself, ``Why {\tt 12}?''

  When you include such things as function arguments, include
  reasonable default values.
}
\end{frame}



\begin{frame}[fragile,c]{Indent!}

\begin{lstlisting}
# move values above/below quantiles to those quantiles
winsorize <-
function(vec, q=0.006)
{
lohi <- quantile(vec, c(q, 1-q), na.rm=TRUE)
if(diff(lohi) < 0)
lohi <- rev(lohi)
vec[!is.na(vec) & vec < lohi[1]] <- lohi[1]
vec[!is.na(vec) & vec > lohi[2]] <- lohi[2]
vec
}
\end{lstlisting}

\note{I taught part of a course on statistical programming at Johns
  Hopkins. It included a lecture like this one. Still, many students
  submitted their programming assignment flush left.

  Most people recommend indenting by 4 spaces. Still, I prefer 2.\\
  (4 spaces are easier to track.)

  Don't use tabs, as they get messed up when moving between
  computers.
}
\end{frame}



\begin{frame}[fragile,c]{Use white space}

\begin{lstlisting}
# move values above/below quantiles to those quantiles
winsorize<-function(vec,q=0.006)
{lohi<-quantile(vec,c(q,1-q),na.rm=TRUE)
if(diff(lohi)<0)lohi<-rev(lohi)
vec[!is.na(vec)&vec<lohi[1]]<-lohi[1]
vec[!is.na(vec)&vec>lohi[2]]<-lohi[2]
vec}
\end{lstlisting}

\note{The computer doesn't care about white space, but people do.
}
\end{frame}


\begin{frame}[fragile,c]{Don't let lines get too long}

\begin{lstlisting}
get_grid_index <-
function(vec, step)
{
  grid <- seq(min(vec), max(vec), by=step)
  index <- match(grid, vec)

  if(any(is.na(index)))
    index <- sapply(grid, function(a,b) { d <- abs(a-b); sampleone(which(d == min(d))) }, vec)

  index
}
\end{lstlisting}


\note{$<$ 72 characters per line

}
\end{frame}


\begin{frame}[fragile,c]{Use parentheses to avoid ambiguity}


\begin{lstlisting}
if( (ndraws1==1) && (ndraws2>1) ) {

 ...

}

leftval <- which( (map - start) <=0 )
\end{lstlisting}


\note{Even if they aren't {\nhilit necessary}, they can be helpful
  to you (or others) later.
}
\end{frame}




\begin{frame}{Names: meaningful}

\bbi
\item Make names descriptive but concise
\item Avoid {\tt tmp1}, {\tt tmp2}, ...
\item Only use {\tt i}, {\tt j}, {\tt x}, {\tt y} in the simplest
  situations
\item If a function is named {\tt fv}, what might it do?
\item If an object is called {\tt nms}, what could it be?
\item Functions as verbs; objects as nouns
\ei

\note{I'm terrible at coming up with good names. But it's really
  important.

  Descriptive names make code self-documenting.

  If functions are named by what they do, you may not need to explain
  further.

  If the object names are meaningful, the code will be readable as
  it stands. If the objects are named {\tt g}, {\tt g2}, and {\tt
  gg}, it'll need a lot of comments.

  I admit, I'm a terrible offender in this regard, especially when I'm
  coding in a hurry. Do as I say, not as I do.
}
\end{frame}




\begin{frame}{Names: consistent}

\bbi
\item {\tt markers} vs {\tt mnames}

\item {\tt camelCase} vs. {\tt pothole\_case}

\item {\tt nind} vs {\tt n.var}

\item If a function/object has one of these, there shouldn't be a
  function/object with the other.

\ei

\note{Consistency makes the names easier to remember.

  If the names are easier to remember, you're less likely to use the
  wrong name and so introduce a bug.

  And, of course, other users will appreciate the consistency.
}
\end{frame}



\begin{frame}{Names: avoid confusion}

\bbi
\item Don't use both {\tt total} and {\tt totals}
\item Don't use both {\tt n.cluster} and {\tt n.clusters}
\item Don't use both {\tt result} and {\tt results}
\item Don't use both {\tt Mat} and {\tt mat}
\item Don't use both {\tt g} and {\tt gg}
\ei

\note{Don't use similar but different names, in the same function or
  across functions.

  It's confusing, and it's prone to bugs from mistyping.
}
\end{frame}











\begin{frame}{Comments}

\bbi
\item Comment the tricky bits and the major sections
\item Don't belabor the obvious
\item Don't comment bad code; rewrite it
\item Document the input/output and purpose, not the mechanics
\item Don't contradict the code
  \bi
  \item this happens if you revise the code but don't revise the
    related comments
  \ei
\item Comment code as you are writing it (or before)
\item Plan to spend 1/4 of your time commenting
\ei


\note{Some things are tricky and deserve explanation, so that when you
  come back to it, you have a guide.

  But in many cases, code can be rewritten to be clear and
  self-documenting.

  Commenting takes time. And no one ever goes back and adds comments
  later. It's another one of those ``invest for the future'' type of
  things.
}
\end{frame}




\begin{frame}{Error/warning messages}

\vspace{6pt}

{\small
\bi
\item Explain what's wrong (and where)
  \bi
  \item {\ttsm error("nrow(X) != nrow(Y)")}
  \ei

\item Suggest corrective action
  \bi
  \item {\ttsm "You need to first run calc.genoprob()."}
  \ei

\item Give details
  \bi
  \item {\ttfn {\hilit "nrow(X) ("}, nrX, {\hilit ") != nrow(Y) ("}, nrY, {\hilit ")"}}
  \ei


\item Don't give error/warning messages that users won't understand.
  \bi
  \item {\ttsm X'X is singular.}
  \ei

\item Don't let users do something stupid without warning

\item Include error checking even in personal code.
\ei
}

\note{Meaningful error messages are hard.

      Writing them to give detailed information takes time.

      And of course {\nhilit you} wouldn't be making these errors, right?

      But remember that you, six months from now, will be like a whole new
      person.

}
\end{frame}


\begin{frame}{Check data integrity}

\bbi
\item Check that the input is as expected, or give warnings/errors.

\item Write these in the first pass (though they're dull).
  \bi
  \item You may not remember your assumptions later
  \ei

\item These are useful for {\hilit documenting} the assumptions.
\ei

\note{You can't include enough such checks.

  Hadley Wickham's {\tt assertthat} package is great for this. We'll
  talk about it in a couple of weeks.
}
\end{frame}


\begin{frame}{Program organization}

\bbi
\item Break code into separate files (say 300 lines?)

\item Each file includes related functions

\item Files should be named meaningfully

\item Include a {\hilit brief} comment at the top.
\ei

\note{My R/qtl package has some {\nhilit really} long files, like {\tt
    util.R}, which seemed like a good idea at the time. I have to use
  a lot of {\tt grep} to find things. And there are plenty of
  functions in there that I have totally forgotten about.

  I used to make really long boiler-plate comments at the top of each
  file. That means that you {\nhilit always} have to page down to get
  to the code. Now I try to write really minimal comments.
}
\end{frame}



\begin{frame}[fragile]{Create an R package!}

\bbi
\item Make a personal package with bits of your own code
\item Mine is R/broman, \href{https://github.com/kbroman/broman}{\ttsm github.com/kbroman/broman}
\ei

\vspace{24pt}

\begin{lstlisting}
# qqline corresponding to qqplot
qqline2 <- function(x, y, probs = c(0.25, 0.75), qtype = 7, ...)
{
  stopifnot(length(probs) == 2)
  x <- quantile(x, probs, names=FALSE, type=qtype, na.rm = TRUE)
  y <- quantile(y, probs, names=FALSE, type=qtype, na.rm = TRUE)
  slope <- diff(y)/diff(x)
  int <- y[1L] - slope*x[1L]
  abline(int, slope, ...)
  invisible(c(intercept=int, slope=slope))
}
\end{lstlisting}
\note{R packages are great for encapsulating and distributing R code
  with documentation.

  Every R programmer should have a personal package containing the
  various functions that they use day-to-day.

  We'll talk about writing R packages next week. It's really rather
  easy.

  I give an example here of one function in my R/broman package.
  Some of the earlier examples are in this package.
}
\end{frame}




\begin{frame}{Complex data objects}

\bbi
\item Keep disparate data together in a more complex
structure.
  \bi
  \item lists in R
  \item I also like to hide things in object {\hilit attributes}
  \ei
\item It's easier to pass such objects between functions
\item Consider object-oriented programming
\ei

\note{Complex data objects allow you to think about complex data as a
  {\hilit single thing}. And they make it easier to pass the data into
  a function.

  R has two systems for object-oriented programming: S3 and S4. They
  can make your code easier to use and understand. But they can also
  make things more opaque and harder to understand.
}
\end{frame}





\begin{frame}{Avoiding bugs}

\bbi
\item Learn to type well.
\item Think before you type.
\item Consider commenting before coding.
\item Code defensively
  \bi
  \item Handle cases that "can't happen"
  \ei
\item Code simply and clearly
\item Use modularity to advantage
\item Think through all special cases
\item Don't be in too much of a hurry
\ei

\note{Writing clear code will make it easier to find bugs, but we also
  want to {\hilit avoid} introducing bugs in the first place. These
  are my strategies; They are useful {\hilit when applied}.
}
\end{frame}





\begin{frame}{Basic principles}

\vspace{18pt}

\bi
\item Code that works
    \bi
    \item[] No bugs; efficiency is secondary (or tertiary)
    \ei
\item Readable
    \bi
    \item[] Fixable; extendible
    \ei
\item Reusable
    \bi
    \item[] Modular; reasonably general
    \ei
\item Reproducible
    \bi
    \item[] Re-runnable
    \ei
\item Think before you code
    \bi
    \item[] More thought $\implies$ fewer bugs/re-writes
    \ei
\item Learn from others' code
    \bi
    \item[] R itself; key R packages
    \ei
\ei

\note{It seems useful to repeat these basic principles.
}
\end{frame}


\begin{frame}{Summary}

\bbi
\item Get the correct answers.
\item Find a clear style and stick to it.
\item Plan for the future.
\item Be organized.
\item Don't be too hurried.
\item Learn from others.
\ei

\note{Our primary goal, as scientific programmers, is to get the right
  answer.

  We all write differently. That's okay. But be consistent. If every
  function is named in a different way, it will be that much harder to
  remember.

  Plan for the future: write code that is readable and easy to
  maintain. And write things in a modular and reasonably general way,
  so that you (or others) will be more likely to be able to re-use
  your work.

  Be organized. Well-organized software is easier to understand and
  maintain.

  It's when I'm hurried that I write crap code. It means more work
  later.

  There are a lot of great programmers out there; read their code and
  learn from it.
}
\end{frame}





\end{document}
